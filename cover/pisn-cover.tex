\documentclass{letter}

\address{Department of Physics and Astronomy\\Stony Brook University\\Stony Brook NY 11794\\United States}

\signature{Will M. Farr\\Maya Fishbach\\Jiani Ye\\Daniel Holz}

\begin{document}

\begin{letter}{Prof. Frederic Rasio\\Northwestern University\\Dept of Physics \& Astronomy\\2145 Sheridan Rd\\Evanston IL 60208-3112\\USA}

\opening{Dear Dr. Rasio:}

Please find enclosed a submission for consideration as an ApJ Letter, titled
``A Future Percent-Level Measurement of the Hubble Expansion at Redshift 0.8
With Advanced LIGO.''

In the manuscript, we point out that by using a feature in the mass distribution
of black holes, we are able to make precision cosmological constraints at $z
\simeq 0.8$ and beyond. The mass feature which we use is the sharp reduction in
the observed rate density of binary black hole mergers at $M \sim 45\, M_\odot$,
thought to be the result of pair instability supernovae (although understanding
the physical mechanism isn't crucial to the use of the feature). This
``absorption'' feature allows us to infer redshift, which, when coupled with the
use of gravitational-wave sources as standard sirens, allows us to measure the
distance-redshift relation without the use of a separate distance ladder.

We show that not only will standard sirens provide definitive measurements of
the Hubble constant, but they will also measure the full expansion history out
to z~1, and thereby constrain the dark energy and deviations from LambdaCDM over
the crucial range where the universe's expansion goes from deceleration to
acceleration. This is a direct, absolute, self-calibrated cosmological
measurement, and is therefore qualitatively different from Type Ia SNe, BAO,
strong lensing, or other ways to constrain cosmology at this epoch.

We are not the first to propose using features in a mass distribution to measure
redshifts in a gravitational wave detector (the idea goes back to Chernoff \&
Finn (1993)); but until now all such proposals exploited the narrow range of
merging \emph{binary neutron star} masses.  Such mergers are not suitable for
cosmography in the current era because the reach of present detectors to neutron
star mergers, $\sim 100 \, \mathrm{Mpc}$, is not sufficient for the mass to
redshift meaningfully.  In contrast, Advanced LIGO and Virgo operating at design
sensitivity can detect a merger of two black holes near the pair instability
limit at redshifts $z \simeq 1.5$, so precision cosmography is possible with
current detectors.

We thank you for your careful consideration of this manuscript.

\closing{Sincerely,}

\end{letter}

\end{document}
