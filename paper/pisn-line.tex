\documentclass[modern]{aastex62}

\usepackage{acro}
\usepackage{amsmath}
\usepackage{color}

% Commands
\newcommand{\citationhere}{\textcolor{blue}{CITE}}
\newcommand{\dd}{\mathrm{d}}
\newcommand{\diff}[2]{\frac{\dd #1}{\dd #2}}
\newcommand{\fixme}[1]{\textcolor{red}{#1}}

% Shorthand
\newcommand{\mdet}{m^\mathrm{detector}}
\newcommand{\MMax}{M_\mathrm{max}}
\newcommand{\MMin}{M_\mathrm{min}}
\newcommand{\monedet}{m_1^\mathrm{detector}}
\newcommand{\monesource}{m_1^\mathrm{source}}
\newcommand{\mtwodet}{m_2^\mathrm{detector}}
\newcommand{\mtwosource}{m_2^\mathrm{source}}
\newcommand{\msource}{m^\mathrm{source}}
\newcommand{\wDE}{w_\mathrm{DE}}

% Quantities
\newcommand{\OOneOTwoAlpha}{0.4^{+1.3}_{-1.9}}
\newcommand{\OOneOTwoMergerRate}{65^{+76}_{-34} \, \perGpcyr}
\newcommand{\MPISN}{45 \, \MSun{}}

% Generated quantities from PaperPlots.ipynb
\newcommand{\MScaleOneYear}{43.0^{+1.3}_{-1.3} \, \MSun{}}
\newcommand{\MScaleFiveYear}{44.64^{+0.76}_{-0.81} \, \MSun{}}
\newcommand{\SigmaHPvtOneYear}{6.1 \%}
\newcommand{\SigmaHPvtFiveYear}{2.9 \%}
\newcommand{\zpivot}{0.8}
\newcommand{\wDEOneYear}{-0.68^{+0.17}_{0.21}}
\newcommand{\wDEFiveYear}{-1.04^{+0.12}_{0.13}}
\newcommand{\SigmawDEFiveYear}{12 \%}

% units
\newcommand{\Gpc}{\mathrm{Gpc}}
\newcommand{\kmsMpc}{\mathrm{km} \, \mathrm{s}^{-1} \, \mathrm{Mpc}^{-1}}
\newcommand{\MSun}{M_\odot}
\newcommand{\perGpcyr}{\mathrm{Gpc}^{-3} \, \mathrm{yr}^{-1}}

% Acronym defn's
\DeclareAcronym{BH}{
  short = BH,
  long = {black hole}
}
\DeclareAcronym{BBH}{
  short = BBH,
  long = {binary black hole}
}
\DeclareAcronym{BNS}{
  short = BNS,
  long = {binary neutron star}
}
\DeclareAcronym{GW}{
  short = GW,
  long = {gravitational wave}
}
\DeclareAcronym{GWTC1}{
  short = GWTC1,
  long = {gravitational wave transient catalog 1}
}
\DeclareAcronym{PISN}{
  short = PISN,
  long = {pair instability supernova}
}

\begin{document}

\title{A Two Percent Measurement }

\author[0000-0003-1540-8562]{Will M. Farr}
\affiliation{Department of Physics and Astronomy, Stony Brook University, Stony Brook NY 11794, USA}
\affiliation{Center for Computational Astronomy, Flatiron Institute, 162 5th Ave., New York NY 10010, USA}
\email{will.farr@stonybrook.edu}

\author[0000-0002-1980-5293]{Maya Fishbach}
\affiliation{Department of Astronomy and Astrophysics, University of Chicago, Chicago IL 60637, USA}
\email{mfishbach@uchicago.edu}

\author{Jiani Ye}
\affiliation{Department of Physics and Astronomy, Stony Brook University, Stony Brook NY 11794, USA}
\email{jiani.ye@stonybrook.edu}

\author[0000-0002-0175-5064]{Daniel E. Holz}
\affiliation{Enrico Fermi Institute, Department of Physics, Department of Astronomy and Astrophysics,\\and Kavli Institute for Cosmological Physics, University of Chicago, Chicago IL 60637, USA}
\email{holz@uchicago.edu}

\begin{abstract}
%
  Simultaneous measurements of distance and redshift can be used to constrain
  the expansion history of the universe and the associated cosmological
  parameters \citationhere{}. Merging \ac{BBH} systems are standard sirens
  \citep{Schutz1986,Holz2005}---their gravitational waveform provides direct
  information about the luminosity distance to the source.  Because gravity is
  scale-free, there is a perfect degeneracy between the source masses and
  redshift; some non-gravitational information is necessary to break the
  degeneracy and determine the redshift of the source
  \citep{Schutz1986,Chernoff1993,Finn1996,Wang1997,Holz2005,Dalal2006,Taylor2012,Messenger2012,GW170817-H0}.
  Here we suggest that the \ac{PISN}
  \citep{Heger2002,Belczynski2016,Woosley2017,Spera2017} process, thought to be
  the source of the observed upper-limit on the \ac{BH} mass in merging \ac{BBH}
  systems at $\sim \MPISN{}$ \citep{O1O2Population}, imprints a mass scale
  in the population of \ac{BBH} mergers and permits a measurement of the
  redshift-luminosity-distance relation.  With realistic assumptions about the
  \ac{BBH} merger rate \citep{GWTC-1}, a mass distribution incorporating a
  \ac{PISN} mass scale \citep{O1O2Population}, and measurement uncertainty
  \citep{Vitale2017} for \ac{BBH} inspirals in the Advanced LIGO and Virgo
  detectors at design sensitivity \citationhere{}, we simulate five years of
  \ac{BBH} detections. We show that after one year of operation at design
  sensitivity (\fixme{put date here}) the \ac{BBH} population can constrain
  $H(z)$ to $\SigmaHPvtOneYear$ at a pivot redshift $z \simeq \zpivot$.  After
  five years (\fixme{date}) the constraint improves to $\SigmaHPvtFiveYear$.
  This measurement relies only on general relativity and the presence of a
  cutoff mass scale that is approximately fixed or calibrated across cosmic time
  ($\Delta \MMax \lesssim 1 \, M_\odot$); it is independent of any distance
  ladder or cosmological model. When combined with a percent-level local
  measurement of the Hubble constant \citep{Chen2017} and a sub-percent
  constraint on the physical matter density from CMB measurements
  \citep{Planck2016} in a $w\mathrm{CDM}$ cosmological model, the dark energy
  equation of state parameter is determined with $\SigmawDEFiveYear$
  uncertainty. Observations by future ``third-generation'' \ac{GW} detectors
  \citationhere{}, which can see \ac{BBH} mergers throughout the universe, would
  permit sub-percent cosmographical measurements to $z \gtrsim 4$ within one
  month of observation.
%
\end{abstract}

\section*{ }

The \ac{GWTC1} contains ten binary black hole merger events observed during
Advanced LIGO and Advanced VIRGO's first and second observing runs
\citep{GWTC-1}. Modeling of this population suggests a precipitous drop in the
merger rate for primary black hole masses larger than $\sim \MPISN{}$
\citep{Fishbach2017,GWTC-1}.  A possible explanation for this drop is the
\ac{PISN} process \citationhere{}. This process occurs in the cores of massive
stars (helium core masses $30$--$133 \, \MSun$ \citep{Woosley2017}) when the
core temperature becomes sufficiently high to permit the production of
electron-positron pairs; pair production softens the equation of state of the
core, leading to a collapse which is halted by nuclear burning
\citep{Heger2002}.  The energy produced can either unbind the star, leaving no
\ac{BH} remnant, or drive a mass-loss pulse that reduces the mass of the star
until the \ac{PISN} is halted, leading to remnant masses $\sim \MPISN{}$.
Modeling suggests that the upper limit on the remnant mass may vary by less than
\fixme{XX} with redshift for $0 \leq z \lesssim 2$ \citep{Belczynski2016}.

Compact object mergers that emit gravitational waves have a universal
characteristic peak luminosity $c^5/G \simeq 3.6 \times 10^{59} \, \mathrm{erg}
\, \mathrm{s}^{-1}$ that enables direct measurements of the luminosity distance
to these sources \citep{Schutz1986}.  They are standard sirens \citep{Holz2005}.
However, the source-frame mass of the merging objects is degenerate with the
redshift; the waveform depends only on the redshifted mass in the detector
frame, $m_\mathrm{det} = m_\mathrm{source} (1 + z)$.  General relativity
predicts the gravitational waveforms of stellar-mass \ac{BBH} mergers.  Using
parameterized models of these waveforms
\citep{Taracchini2014,Kahn2016,Bohe2017,Chatziioannou2017}, it will be possible
to measure the detector-frame masses with $\sim 20\%$ uncertainty and luminosity
distances \citep{Hogg1999} with $\sim 50\%$ uncertainty for a source near the
detection threshold in Advanced LIGO and Anvanced Virgo at design sensitivity
\citep{Vitale2017} \citationhere{}.  The uncertainty in these parameters scales
inversely with the signal-to-noise ratio of a source \citep{Vitale2017}.

If the \ac{BBH} merger rate follows the star formation rate
\citep{Fishbach2018,O1O2Population}, the primary mass distribution follows a
declining power law $m_1^{-\alpha}$ with $\alpha \simeq 0.75$ for $m_1 \lesssim
\MPISN{}$, the mass ratio distribution is flat, and the three-detector duty
cycle is $\sim 50\%$ then Advanced LIGO and Advanced Virgo should detect $\sim
1000$ \ac{BBH} mergers per year at design sensitivity over a range of redshifts
$0 \leq z \lesssim 1.5$.  The typical detected merger will have a redshift $z
\sim 0.75$.  If we assume that $\sim 1/2$ these detections are informative about
the redshifted upper limit on the remnant mass in the detector frame,
$m_\mathrm{max,det} = m_\mathrm{max,source} \left(1 + z\left( d_L \right)
\right)$, that the combined uncertainty is $1/\sqrt{N}$ smaller than the
single-measurement uncertainty, and that most detections are near threshold,
then we are dominated by the $\sim 50\%$ distance uncertainty and can achieve an
absolute distance-redshift measurement (i.e.\ constrain the local expansion
rate, $H(z)$, for $z \simeq 0.75$) at the $50 \% / \sqrt{1000/2} \simeq 2.2 \%$
level after one year, and the $1 \%$ level after five years of \ac{BBH} merger
observations at design sensitivity.

Detailed calculations support this back-of-the-envelope estimate.  We have
simulated five years of \ac{GW} observations with Advanced LIGO and Advanced
Virgo at design sensitivity.  We use a local merger rate, mass distribution, and
rate evolution with redshift that are consistent with current observations
\citep{Fishbach2017,Fishbach2018,O1O2Population}.  Our mass distribution
includes a sharp cutoff at $m = \MPISN{}$ to model the effects of the
\ac{PISN} process \citep{Belczynski2016}.  We use a realistic model of the
detectability of sources from this population
\citep{GW150914Rate,GW150914RateSupplement} and for mass and distance estimation
uncertainties \citep{Vitale2017}.  The properties of the simulated population
are described more fully in \S \ref{sec:simulated-population}.

\begin{figure}
  \plottwo{plots/m1-dL-true}{plots/m1-dL-obs}
%
  \caption{\label{fig:m1-dL} Simulated population of \ac{BBH} mergers.  (Left)
  The true detector-frame primary \ac{BH} masses and luminosity distances for
  the simulated population of \ac{BBH} mergers used in this work.  Blue circles
  denote one year of Advanced LIGO / Virgo observations, orange circles five
  years of observations. The black line shows the redshifting of the maximum
  \ac{BH} mass corresponding to the cosmology used to generate the events
  \citep[TT, TE, EE + lowP + lensing + ext]{Planck2016}.  (Right) The inferred
  detector-frame primary \ac{BH} masses and luminosity distances using our model
  for the \ac{GW} data likelihood for each \ac{BBH} event.  Here again blue
  corresponds to one year of observations and orange to five years.  Dots denote
  the mean and bars the 1$\sigma$ width of the likelihood (i.e.\ a posterior
  distribution using flat priors on $m_1$ and $d_L$).  There is a bias in the
  recovery of the masses and distance that becomes more acute at large distances
  due to a failure to model the population (which is not flat in $m_1$ and
  $d_L$) and selection effects in these single-event analyses.  The black line
  shows the redshifting of the mass upper limit.  The most-distant event biases
  upward in both mass and distance by several sigma because it represents a
  single ``lucky'' noise fluctuation into detectability out of $\sim 2 \times
  10^5$ merger events per year within the detector horizon.  We also show the
  inferred distance-maximum detector-frame mass relation inferred from our
  analysis of the one year and five year mock data sets in the same colors (the
  solid line gives the posterior median, dark band gives the 68\% credible
  interval, and the light band the 95\% credible interval).}
%
\end{figure}

\begin{figure}
  \plottwo{plots/Hz}{plots/H075}
%
  \caption{\label{fig:Hz} Inferred cosmological expansion history and distance
  scale.  (Left) The local expansion rate, $H(z)$, inferred from an analysis of
  the one year (blue) and five year (orange) simulated populations using a mass
  distribution model with a parameterized cutoff mass (see text).  The black
  line gives the cosmology used to generate the simulated population \citep[TT,
  TE, EE + lowP + lensing + ext]{Planck2016}.  The solid lines give the
  posterior median $H(z)$ at each redshift; the bands give 1$\sigma$ (68\%) and
  2$\sigma$ (95\%) credible intervals.  The 1$\sigma$ fractional uncertainty on
  $H(z)$ is minimized at $z \simeq \zpivot{}$ for both data sets; after one year
  it is \SigmaHPvtOneYear{} and after five years it is \SigmaHPvtFiveYear{}.
  (Right) Posterior distributions over $H\left(z = \zpivot{}\right)$,
  corresponding to the redshift where the fractional uncertainty is minimized.
  The true $H\left( z = \zpivot{} \right)$ is shown by the black vertical line.
  The posterior after one year is blue, after five years is orange.  This
  demonstrates an absolute distance measure to $z \simeq \zpivot{}$ at
  percent-level precision; combining this inference on $H(z)$ with other data
  sets such as observations of baryon acoustic oscillations \citep{BOSS2015} or
  Type Ia supernovae \citep{Scolnic2018} can translate this absolute distance
  measure to other redshifts \citep{BOSS2015,Cuesta2015,Feeney2019}. For
  example, one can independently calibrate the Type Ia supernova distance scale
  without a distance ladder \citep{Feeney2019,Scolnic2018}, or compare the
  \ac{GW}-determined distance scale with one derived from the photon-baryon
  sound horizon \citep{Cuesta2015,Aylor2019} in the early universe
  \citep{Planck2016} or at late times \citep{BOSS2015}. }
%
\end{figure}

\begin{figure}
  \plotone{plots/mass-correction}
%
  \caption{\label{fig:mass-correction} Inferred maximum mass and masses and
  redshifts for one year of observation.  Posterior mean and 1$\sigma$ (68\%)
  credible ranges for the source-frame primary \ac{BH} mass and redshift after
  one year of \ac{BBH} merger observations.  The horizontal line is the
  posterior median of the maximum black hole mass set by the PISN process; the
  dark and light bands correspond to the 1$\sigma$ and 2$\sigma$ (68\% and 95\%)
  credible intervals on the maximum mass.  (Compare to Figure \ref{fig:m1-dL}.)
  Our model adjusts cosmological parameters, and therefore the correspondence
  between the measured detector-frame masses and luminosity distances and
  inferred redshifts and source-frame masses, until it achieves a consistent
  upper limit on the source-frame \ac{BH} mass across all redshifts.  After one
  year of synthetic observations we measure $\MMax{} = \MScaleOneYear{}$ (median
  and 68\% credible interval).  After five years (not shown) we measure $\MMax{}
  = \MScaleFiveYear{}$.}
%
\end{figure}

\begin{figure}
  \plotone{plots/wDE}
%
  \caption{\label{fig:wDE} Posterior on the dark energy equation of state
  parameter after imposing additional cosmological constraints.  If we impose a
  1\% measurement of $H_0$ \citep{Chen2017,Mortlock2018} and the constraints on
  $\Omega_M h^2$ from existing observations of the cosmic microwave background
  \citep{Planck2016}, we can infer the equation of state parameter $\wDE{}
  \equiv P_\mathrm{DE} / \rho_\mathrm{DE}$ for dark energy in a $w$CDM
  cosmological model.  (We do not obtain any meaningful constraint on the
  evolution of $\wDE{}$ with redshift when this parameter is allowed to vary, so
  we fix it to a constant across all redshifts.)  We use $\wDE{} = -1$ to
  generate our data set; this value is indicated by the black line above.  The
  posterior obtained on $\wDE{}$ after one year of synthetic observations is
  shown in blue and after five years in orange.  We find $\wDE{} =
  \wDEOneYear{}$ after one year (median and 68\% credible interval) and $\wDE{}
  = \wDEFiveYear{}$ after five years.}
%
\end{figure}

\acknowledgments

We thank Stephen Feeney for providing a sounding board for the methods discussed
in this paper.  We acknowledge the 2018 April APS Meeting and Barley's Brewing
Company in Columbus, OH, USA where this work was originally conceived.

\bibliography{pisn-line}

\appendix

\section{Simulated Population}
\label{sec:simulated-population}

Blah.

\end{document}
