\documentclass[modern]{aastex62}

\usepackage{acro}
\usepackage{amsmath}
\usepackage{color}

% Commands
\newcommand{\citationhere}{\textcolor{blue}{CITE}}
\newcommand{\dd}{\mathrm{d}}
\newcommand{\diff}[2]{\frac{\dd #1}{\dd #2}}
\newcommand{\fixme}[1]{\textcolor{red}{#1}}
\newcommand{\fsm}{f_{\mathrm{smooth}}}

% Shorthand
\newcommand{\mdet}{m^\mathrm{detector}}
\newcommand{\MScale}{M_{\mathrm{scale}}}
\newcommand{\monedet}{m_1^\mathrm{detector}}
\newcommand{\monesource}{m_1^\mathrm{source}}
\newcommand{\mtwodet}{m_2^\mathrm{detector}}
\newcommand{\mtwosource}{m_2^\mathrm{source}}
\newcommand{\msource}{m^\mathrm{source}}
\newcommand{\wDE}{w_\mathrm{DE}}

% Quantities
\newcommand{\OOneOTwoAlpha}{0.4^{+1.3}_{-1.9}}
\newcommand{\OOneOTwoMergerRate}{65^{+76}_{-34} \, \perGpcyr}
\newcommand{\MPISN}{45 \, \MSun{}}
\newcommand{\MTaperScale}{5 \, \MSun{}}

% Generated quantities from PaperPlots.ipynb
\newcommand{\MScaleOneYear}{43.0^{+1.3}_{-1.3} \, \MSun{}}
\newcommand{\MScaleFiveYear}{44.64^{+0.76}_{-0.81} \, \MSun{}}
\newcommand{\SigmaHPvtOneYear}{6.1 \%}
\newcommand{\SigmaHPvtFiveYear}{2.9 \%}
\newcommand{\SigmaHNaughtTransFiveYear}{2.0}
\newcommand{\zpivot}{0.8}
\newcommand{\wDEOneYear}{-0.68^{+0.17}_{0.21}}
\newcommand{\wDEFiveYear}{-1.04^{+0.12}_{0.13}}
\newcommand{\SigmawDEOneYear}{19 \%}
\newcommand{\SigmawDEFiveYear}{12 \%}

% units
\newcommand{\Gpc}{\mathrm{Gpc}}
\newcommand{\kmsMpc}{\mathrm{km} \, \mathrm{s}^{-1} \, \mathrm{Mpc}^{-1}}
\newcommand{\MSun}{M_\odot}
\newcommand{\perGpcyr}{\mathrm{Gpc}^{-3} \, \mathrm{yr}^{-1}}

% Acronym defn's
\DeclareAcronym{BH}{
  short = BH,
  long = {black hole}
}
\DeclareAcronym{BBH}{
  short = BBH,
  long = {binary black hole}
}
\DeclareAcronym{BNS}{
  short = BNS,
  long = {binary neutron star}
}
\DeclareAcronym{CMB}{
  short = CMB,
  long = {cosmic microwave background}
}
\DeclareAcronym{GW}{
  short = GW,
  long = {gravitational wave}
}
\DeclareAcronym{GWTC1}{
  short = GWTC1,
  long = {gravitational wave transient catalog 1}
}
\DeclareAcronym{PISN}{
  short = PISN,
  long = {pair instability supernova}
}
\DeclareAcronym{PPISN}{
  short = PPISN,
  long = {pulsational pair instability supernova}
}

\begin{document}

\title{A Future Percent-Level Measurement of the Hubble Parameter at Redshift 0.8 With
  Gravitational Waves}

\author[0000-0003-1540-8562]{Will M. Farr}
\affiliation{Department of Physics and Astronomy, Stony Brook University, Stony Brook NY 11794, USA}
\affiliation{Center for Computational Astronomy, Flatiron Institute, 162 5th Ave., New York NY 10010, USA}
\email{will.farr@stonybrook.edu}

\author[0000-0002-1980-5293]{Maya Fishbach}
\affiliation{Department of Astronomy and Astrophysics, University of Chicago, Chicago IL 60637, USA}
\email{mfishbach@uchicago.edu}

\author{Jiani Ye}
\affiliation{Department of Physics and Astronomy, Stony Brook University, Stony Brook NY 11794, USA}
\email{jiani.ye@stonybrook.edu}

\author[0000-0002-0175-5064]{Daniel E. Holz}
\affiliation{Enrico Fermi Institute, Department of Physics, Department of Astronomy and Astrophysics,\\and Kavli Institute for Cosmological Physics, University of Chicago, Chicago IL 60637, USA}
\email{holz@uchicago.edu}

\begin{abstract}
%
  Simultaneous measurements of distance and redshift can be used to constrain
  the expansion history of the universe and associated cosmological parameters.
  Merging \ac{BBH} systems are standard sirens
  \citep{Schutz1986,Holz2005}---their gravitational waveform provides direct
  information about the luminosity distance to the source.  Because gravity is
  scale-free, there is a perfect degeneracy between the source masses and
  redshift; some non-gravitational information is necessary to break the
  degeneracy and determine the redshift of the source
  \citep{Schutz1986,Chernoff1993,Finn1996,Wang1997,Holz2005,Dalal2006,Taylor2012,Messenger2012,GW170817-H0,Fishbach2019,Soares-Santos2019}.
  Here we suggest that the \ac{PISN} process
  \citep{Fowler1964,Rakavy1967,Bond1984,Heger2002,Belczynski2016,Woosley2017,Spera2017},
  thought to be the source of the observed upper-limit on the \ac{BH} mass in
  merging \ac{BBH} systems at $\sim \MPISN{}$ \citep{O1O2Population}, imprints a
  mass scale in the population of \ac{BBH} mergers and permits a measurement of
  the redshift-luminosity-distance relation with these sources.  We simulate
  five years of \ac{BBH} detections in the Advanced LIGO and Virgo detectors
  with realistic assumptions about the \ac{BBH} merger rate \citep{GWTC-1}, a
  mass distribution incorporating a smooth \ac{PISN} cutoff
  \citep{O1O2Population}, and measurement uncertainty \citep{Vitale2017}. We
  show that after one year of operation at design sensitivity (circa 2021
  \citep{ObsScenarios}) the \ac{BBH} population can constrain $H(z)$ to
  $\SigmaHPvtOneYear$ at a pivot redshift $z \simeq \zpivot$.  After five years
  (circa 2025) the constraint improves to $\SigmaHPvtFiveYear$. If the \ac{PISN}
  cutoff is sharp, the uncertainty is smaller by about a factor of two.  This
  measurement relies only on general relativity and the presence of a cutoff
  mass scale that is approximately fixed or calibrated across cosmic time; it is
  independent of any distance ladder or cosmological model. When combined with a
  percent-level local measurement of the Hubble constant \citep{Chen2017} and a
  sub-percent constraint on the physical matter density from CMB measurements
  \citep{Planck2016} in a $w\mathrm{CDM}$ cosmological model, the dark energy
  equation of state parameter is determined with $\SigmawDEFiveYear$
  uncertainty. Observations by future ``third-generation'' \ac{GW} detectors
  \citep{Punturo2010,CosmicExplorer}, which can see \ac{BBH} mergers throughout
  the universe, would permit sub-percent cosmographical measurements to $z
  \gtrsim 4$ within one month of observation.
%
\end{abstract}

\section*{ }

The \ac{GWTC1} contains ten binary black hole merger events observed during
Advanced LIGO and Advanced VIRGO's first and second observing runs
\citep{GWTC-1}. Modeling of this population suggests a precipitous drop in the
merger rate for primary black hole masses larger than $\sim \MPISN{}$
\citep{Fishbach2017,GWTC-1}.  A possible explanation for this drop is the
\ac{PISN} process
\citep{Fowler1964,Rakavy1967,Bond1984,Heger2002,Belczynski2016,Woosley2017,Spera2017}.
This process occurs in the cores of massive stars (helium core masses $30$--$133
\, \MSun$ \citep{Woosley2017}) when the core temperature becomes sufficiently
high to permit the production of electron-positron pairs; pair production
softens the equation of state of the core, leading to a collapse which is halted
by nuclear burning \citep{Heger2002}.  The energy produced can either unbind the
star, leaving no \ac{BH} remnant, or drive a mass-loss pulse that reduces the
mass of the star until the \ac{PISN} is halted, leading to remnant masses $\sim
\MPISN{}$ (this latter process is called the \ac{PPISN}). The characteristic
mass of remnant black holes depends weakly on the metallicity of the progenitor
stars; modeling suggests that the upper limit on the remnant mass may vary by
less than 1--2 $\MSun$ for redshifts $0 \leq z \lesssim 2$
\citep{Belczynski2016,Mapelli2017}.  Here we make the conservative choice to
model the effcet as a smooth taper in the mass distribution that takes effect
around $m \simeq \MPISN{}$ but acts over a characteristic scale of $\simeq
\MTaperScale$ (see \S\ \ref{sec:simulated-population} for a full description of
our model).

Compact object mergers that emit gravitational waves have a universal
characteristic peak luminosity $c^5/G \simeq 3.6 \times 10^{59} \, \mathrm{erg}
\, \mathrm{s}^{-1}$ that enables direct measurements of the luminosity distance
to these sources \citep{Schutz1986}.  They are ``standard sirens''
\citep{Holz2005}. However, the effects of the source-frame mass and redshift are
degenerate in the gravitational waveform; the observed waveform depends only on
the redshifted mass in the detector frame, $m_\mathrm{det} = m_\mathrm{source}
(1 + z)$. General relativity predicts the gravitational waveforms of
stellar-mass \ac{BBH} mergers.  Using parameterized models of these waveforms
\citep{Taracchini2014,Kahn2016,Bohe2017,Chatziioannou2017}, it will be possible
to measure the detector-frame masses with $\sim 20\%$ uncertainty and luminosity
distances \citep{Hogg1999} with $\sim 50\%$ uncertainty for a source near the
detection threshold in Advanced LIGO and Anvanced Virgo at design sensitivity
\citep{Vitale2017}.  The relative uncertainty in these parameters scales
inversely with the signal-to-noise ratio of a source.

If the \ac{BBH} merger rate follows the star formation rate
\citep{Fishbach2018,O1O2Population}, the primary mass distribution follows a
declining power law $m_1^{-\alpha}$ with $\alpha \simeq 0.75$ for $m_1 \lesssim
\MPISN{}$, tapering off above this mass scale, the mass ratio distribution is
flat, and the three-detector duty cycle is $\sim 50\%$ then Advanced LIGO and
Advanced Virgo should detect $\sim 1000$ \ac{BBH} mergers per year at design
sensitivity over a range of redshifts $0 \leq z \lesssim 1.5$.  The typical
detected merger will have a redshift $z \sim 0.5$.  If we assume that $\sim
1/4$ these detections are informative about the redshifted upper limit on the
remnant mass in the detector frame, $m_\mathrm{max,det} = m_\mathrm{max,source}
\left(1 + z\left( d_L \right) \right)$, that the combined uncertainty is
$1/\sqrt{N}$ smaller than the single-measurement uncertainty, and that most
detections are near threshold, then we are dominated by the $\sim 50\%$ distance
uncertainty and can achieve an absolute distance-redshift measurement (i.e.\
constrain the local expansion rate, $H(z)$, for $z \simeq \zpivot{}$) at the $50 \% /
\sqrt{1000/4} \simeq 3 \%$ level after one year, and the $1.4 \%$ level after
five years of \ac{BBH} merger observations at design sensitivity.

Detailed calculations are within a factor of two of this back-of-the-envelope
estimate.  We have simulated five years of \ac{GW} observations with Advanced
LIGO and Advanced Virgo at design sensitivity.  We use a local merger rate, mass
distribution, and rate evolution with redshift that are consistent with current
observations \citep{Fishbach2017,Fishbach2018,O1O2Population}.  Our mass
distribution tapers off at $m = \MPISN{}$ to model the effects of the \ac{PISN}
process \citep{Belczynski2016}.  We use a realistic model of the detectability
of sources from this population \citep{GW150914Rate,GW150914RateSupplement} and
for mass and distance estimation uncertainties \citep{Vitale2017}.  The
properties of the simulated population are described more fully in \S
\ref{sec:simulated-population}.  Figure \ref{fig:m1-dL} shows the simulated
detections and uncertainty on detector-frame mass and distance estimates for one
and five years of observation.

\begin{figure}
  \plottwo{plots/m1-dL-true}{plots/m1-dL-obs}
%
  \caption{\label{fig:m1-dL} Simulated population of \ac{BBH} mergers.  (Left)
  The true detector-frame primary \ac{BH} masses and luminosity distances for
  the simulated population of \ac{BBH} mergers used in this work.  Blue circles
  denote one year of Advanced LIGO / Virgo observations, orange circles five
  years of observations. The black line shows the redshifting of the PISN
  \ac{BH} mass scale corresponding to the cosmology used to generate the events
  \citep[TT, TE, EE + lowP + lensing + ext]{Planck2016}.  (Right) The inferred
  detector-frame primary \ac{BH} masses and luminosity distances using our model
  of the measurement uncertainty for each event.  Here again blue corresponds to
  one year of observations and orange to five years.  Dots denote the mean and
  bars the 1$\sigma$ width of the likelihood for each event.  There is a bias in
  the recovery of the masses and distance that becomes more acute at large
  distances due to a failure to model the population (which is not flat in $m_1$
  and $d_L$) and selection effects in these single-event analyses.  The black
  line shows the redshifting of the \ac{PISN} mass scale.  The most-distant
  event biases upward in both mass and distance by several sigma because it
  represents a single ``lucky'' noise fluctuation into detectability out of
  $\sim 2 \times 10^5$ merger events per year within the detector horizon.  We
  also show the inferred distance-mass relation from our analysis of the one
  year and five year mock data sets in the same colors (the solid line gives the
  posterior median, dark band gives the 68\% credible interval, and the light
  band the 95\% credible interval).}
%
\end{figure}

We fit a parameterized model of the true mass distribution to this data set
accounting for measurement error and selection effects in a hierarchical
analysis \citep{Hogg2010,Mandel2010,Loredo2004,Mandel2019,Farr2019}.  We include
parameters for the power-law slopes in the mass distribution and redshift
evolution, a mass scale and range of masses over which the mass distribution
cuts off due to the \ac{PISN}, a parameterized FLRW cosmology with $H_0$,
$\Omega_M$, and $w$ free parameters \citep{Hogg1999}, and parameters for the
true masses and redshift of each detected signal.  The ``population-level''
distribution and cosmological parameters are given broad priors that are much
wider than the corresponding posteriors.  Marginalizing over all parameters
except $H_0$, $\Omega_M$ and $w$ induces a posterior over expansion histories,
$H(z)$, that is shown in Figure \ref{fig:Hz}.  The redshift at which the
fractional uncertainty in $H(z)$ is minimized---the ``pivot'' redshift---is
$\zpivot{}$.  After one year of observations, the fractional uncertainty in $H(z
= \zpivot)$ is $\SigmaHPvtOneYear{}$; after five years it is
$\SigmaHPvtFiveYear{}$.  This demonstrates an absolute distance measure to $z
\simeq \zpivot{}$ at percent-level precision; combining this inference on $H(z)$
with other data sets such as observations of baryon acoustic oscillations
\citep{BOSS2015} or Type Ia supernovae \citep{Scolnic2018} can translate this
absolute distance measure to other redshifts (at $z = 0$ it would correspond to
an uncertainty on $H_0$ of $\pm \SigmaHNaughtTransFiveYear{} \, \kmsMpc$)
\citep{BOSS2015,Cuesta2015,Feeney2019}. For example, one can independently
calibrate the Type Ia supernova distance scale without a distance ladder
\citep{Feeney2019,Scolnic2018}, or compare the \ac{GW}-determined distance scale
with one derived from the photon-baryon sound horizon
\citep{Cuesta2015,Aylor2019} in the early universe \citep{Planck2016} or at late
times \citep{BOSS2015}.

\begin{figure}
  \plottwo{plots/Hz}{plots/H08}
%
  \caption{\label{fig:Hz} Inferred cosmological expansion history and distance
  scale.  (Left) The local expansion rate, $H(z)$, inferred from an analysis of
  the one year (blue) and five year (orange) simulated populations using a mass
  distribution model with a parameterized cutoff mass (see text).  The black
  line gives the cosmology used to generate the simulated population \citep[TT,
  TE, EE + lowP + lensing + ext]{Planck2016}.  The solid lines give the
  posterior median $H(z)$ at each redshift; the bands give 1$\sigma$ (68\%) and
  2$\sigma$ (95\%) credible intervals.  The 1$\sigma$ fractional uncertainty on
  $H(z)$ is minimized at $z \simeq \zpivot{}$ for both data sets; after one year
  it is \SigmaHPvtOneYear{} and after five years it is \SigmaHPvtFiveYear{}.
  (Right) Posterior distributions over $H\left(z = \zpivot{}\right)$,
  corresponding to the redshift where the fractional uncertainty is minimized.
  The true $H\left( z = \zpivot{} \right)$ is shown by the black vertical line.
  The posterior after one year is blue, after five years is orange. }
%
\end{figure}

Our hierarchical model also estimates the source-frame masses and redshifts for
each individual event that incorporate our information about the population.
These results for the one-year data set are shown in Figure
\ref{fig:mass-correction}.  Events pile up near the \ac{PISN} mass scale; in
effect, the cosmology is adjusted so that the measured distances to each event
generate redshifts that produce a constant \ac{PISN} mass scale in the
source-frame from measured detector-frame masses.

\begin{figure}
  \plotone{plots/mass-correction}
%
  \caption{\label{fig:mass-correction} Inferred maximum mass and masses and
  redshifts for one year of observation.  Posterior mean and 1$\sigma$ (68\%)
  credible ranges for the source-frame primary \ac{BH} mass and redshift after
  one year of \ac{BBH} merger observations.  The horizontal line is the
  posterior median of the maximum black hole mass set by the PISN process; the
  dark and light bands correspond to the 1$\sigma$ and 2$\sigma$ (68\% and 95\%)
  credible intervals on the maximum mass.  (Compare to Figure \ref{fig:m1-dL}.)
  Our model adjusts cosmological parameters, and therefore the correspondence
  between the measured detector-frame masses and luminosity distances and
  inferred redshifts and source-frame masses, until it achieves a consistent
  upper limit on the source-frame \ac{BH} mass across all redshifts.  After one
  year of synthetic observations we measure $\MScale{} = \MScaleOneYear{}$ (median
  and 68\% credible interval).  After five years (not shown) we measure $\MScale{}
  = \MScaleFiveYear{}$.}
%
\end{figure}

The pivot redshift for this measurement is close to the redshift where the
physical matter and dark energy densities are equal, and thus this measurement
can be informative about the dark energy equation of state.  If we assume an
independent 1\% measurement of $H_0$ (as could be obtained from \ac{GW}
observations of \ac{BNS} mergers with identified electromagnetic counterparts
\citep{Chen2017}) and a measurement of the physical matter density at high
redshift (as obtained by the Planck satellite's measurements of the \ac{CMB}
\citep{Planck2016}), then the remaining un-constrained parameter in our
cosmological model is $w$, the dark energy equation of state.  Imposing these
additional measurements as a tight prior on the relevant parameters, we find
that our synthetic population of \ac{BBH} mergers can constrain $w$ to
$\SigmawDEOneYear{}$ and $\SigmawDEFiveYear{}$ after one and five years of
observations.  These measurements would be competitive with, but independent
from, other constraints on $w$ \citationhere{}.  Posteriors for $w$ with these
informative priors are shown in Figure \ref{fig:wDE}.

\begin{figure}
  \plotone{plots/wDE}
%
  \caption{\label{fig:wDE} Posterior on the dark energy equation of state
  parameter after imposing additional cosmological constraints.  If we impose a
  1\% measurement of $H_0$ \citep{Chen2017,Mortlock2018} and the constraints on
  $\Omega_M h^2$ from existing observations of the cosmic microwave background
  \citep{Planck2016}, we can infer the equation of state parameter $\wDE{}
  \equiv P_\mathrm{DE} / \rho_\mathrm{DE}$ for dark energy in a $w$CDM
  cosmological model.  (We do not obtain any meaningful constraint on the
  evolution of $\wDE{}$ with redshift when this parameter is allowed to vary, so
  we fix it to a constant across all redshifts.)  We use $\wDE{} = -1$ to
  generate our data set; this value is indicated by the black line above.  The
  posterior obtained on $\wDE{}$ after one year of synthetic observations is
  shown in blue and after five years in orange.  We find $\wDE{} =
  \wDEOneYear{}$ after one year (median and 68\% credible interval) and $\wDE{}
  = \wDEFiveYear{}$ after five years.}
%
\end{figure}

Our simplistic analysis here assumes that the mass distribution of merging
\acp{BBH} does not change with redshift.  In reality the mass distribution will
change because the metallicity of \ac{BBH} progenitor systems changes with
redshift \citep{Belczynski2016,Mapelli2017}.  The \ac{PISN} mass scale, however,
is not expected to evolve by more than $1$--$2\,\MSun{}$ to $z \simeq 1.5$
\citep{Belczynski2016,Mapelli2017}.  We infer a mass scale in our simple model
of $\MScaleFiveYear{}$ after five years; changes in the \ac{PISN} mass scale for
merging \ac{BBH} systems at a comparable level are a systematic that must be
calibrated to ensure an accurate measurement.  \ac{BBH} mergers thus become
``standardizable sirens''.

There is a possibility that the \ac{PPISN} process, a sequence of incomplete
pair-instability-driven mass loss events, could lead to a pile-up of \ac{BBH}
systems near the upper mass limit
\citep{Belczynski2016,Marchant2018,Talbot2018}.  Current LIGO obesrvations are
inconclusive about the existence of such a ``pile up'' in the mass distribution
\citep{O1O2Population}.  Should one exist, it would offer another mass scale in
the mass distribution that could improve upon the constraints presented here. It
may also be possible to detect and calibrate evolution in the \ac{PISN} limit by
comparing the location and amplitude of the pile up as a function of distance,
since these properties would respond differently to a change in the \ac{PISN}
mass scale.

The possible existence of so-called ``second generation'' \ac{BBH} mergers
(mergers where one black hole it itself a merger product) could fill in the
\ac{PISN} mass gap, but are not expected to be prevalent enough to obscure the
falloff in the mass distribution due to the \ac{PISN} limit discussed here
\cite{Rodriguez2019}.

It is likely that by the mid 2020s there will be two gravitational wave
detectors operating in addition to the two LIGO and one Virgo detectors
\cite{ObsScenarios}; additional detectors do not dramatically improve distance
or mass estimates \citep{Vitale2017}, but the higher SNR afforded from the
additional detectors could extend the detection horizon leading to a factor of
$\sim 4$ increase in the number of \ac{BBH} detections and a resulting factor of
two improvement in the constraints presented here.  Third generation \ac{GW}
detectors, planned for construction in the mid-2030s, would detect $\sim 15,000$
\ac{BBH} mergers per month to $z \gtrsim 10$, with a typical relative
uncertainty on $d_L$ of $\sim 10 \%$ at $z \simeq 2$ \citep{Vitale2018}.
Provided the \ac{PISN} mass scale is properly calibrated, such detectors could
achieve sub-percent uncertainty in cosmography to high redshifts $z \lesssim 5$
within \emph{one month} of \ac{BBH} merger observations.

\acknowledgments

We thank Stephen Feeney for providing a sounding board for the methods discussed
in this paper.  We acknowledge the 2018 April APS Meeting and Barley's Brewing
Company in Columbus, OH, USA where this work was originally conceived.

\bibliography{pisn-line}

\appendix

\setcounter{figure}{0}
\renewcommand{\figurename}{Extended Data Figure}
\renewcommand{\tablename}{Extended Data Table}

\section{Simulated Population}
\label{sec:simulated-population}

We draw our synthetic observations from a population that follows
%
\begin{multline}
  \label{eq:population}
  \diff{N}{m_1 \dd m_2 \dd V \dd t} = \frac{R_{30}}{\left( 30 \, \MSun \right)^2} \left( \frac{m_1}{30 \, \MSun} \right)^{-\alpha} \left( \frac{m_2}{30 \, \MSun} \right)^{\beta} \left( 1 + z \right)^{\gamma} \\ \times \fsm\left( m_1 \mid m_l, \sigma_l, m_h, \sigma_h \right) \fsm\left( m_2 \mid m_l, \sigma_l, m_h, \sigma_h \right),
\end{multline}
%
where all quantities are evaluated in the comoving frame and
%
\begin{equation}
  \label{eq:smooth}
  \fsm\left( m \mid m_l, \sigma_l, m_h, \sigma_h \right) = \Phi\left( \frac{\log m - \log m_l}{\sigma_l} \right) \left[ 1 - \Phi\left( \frac{\log m - \log m_h}{\sigma_h} \right) \right]
\end{equation}
%
is a function that tapers smoothly to zero when $m \lesssim m_l$ or $m \gtrsim
m_h$ over a scale in log-mass of $\sigma_l$ and $\sigma_h$; $\Phi(x)$ is the
standard normal cumulative distribution function.  (We enforce $m_2 \leq m_1$.)

We have chosen population parameters that are consistent with the current
observations reported in GWTC-1 \citep{GWTC-1,O1O2Population}:
%
\begin{eqnarray}
  \label{eq:parameters}
  R_{30} & = & 64.4 \\
  \alpha & = & 0.75 \\
  \beta & = & 0.0 \\
  \gamma & = & 3.0 \\
  m_l & = & 5 \, \MSun \\
  m_h & = & 45 \, \MSun \\
  \sigma_l & = & 0.1 \\
  \sigma_h & = & 0.1.
\end{eqnarray}
%
with these choices the volumetric merger rate at $z = 0$ is $60 \, \perGpcyr$.
The corresponding marginal mass distributions for $m_1$ and $m_2$ are shown in
Figure \ref{fig:marginal-masses}.

\begin{figure}
  \plotone{plots/pm1m2-marg}
%
  \caption{\label{fig:marginal-masses} \textbf{Mass distributions.} The joint
  and  marginal mass distributions for the masses in merging \ac{BBH} systems
  implied by the merger rate density in Eq.\ \eqref{eq:population} and the
  parameter choices in Eq.\ \eqref{eq:parameters}. The turnover at $m \simeq
  \MPISN{}$ due to the \ac{PISN} mass scale is apparent in the primary mass
  distribution.}
%
\end{figure}

\end{document}
